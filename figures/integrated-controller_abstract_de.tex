\documentclass[tikz, 14pt, border=10pt]{standalone}

\usepackage{tikz}
\usetikzlibrary{shapes,arrows,positioning,calc}

\begin{document}

\tikzset{
block/.style = {draw, fill=white, rectangle, minimum height=3em, minimum width=6em, on grid},
tmp/.style  = {coordinate, on grid}, 
sum/.style= {draw, fill=white, circle, inner sep=0pt, on grid},
joint/.style= {inner sep=-2pt, on grid},
input/.style = {coordinate},
output/.style= {coordinate},
pinstyle/.style = {pin edge={to-,thin,black}
}
}

\begin{tikzpicture}[auto,
    node distance=15mm and 35mm,
    >=latex']
%\draw [help lines, step=0.5] (-1,-7) grid (23,7);


%%% Nodes
% Process Nodes
    % Input
    \node [block, align=center] (orientation-input) {Ziel-\\ Orientierung};
    \node at (2, 0) [sum] (orientation-sum) {$+$};
    \node at (3.5, 0) [sum] (pe-sum) {$-$};
    % Platform Controller
    \node [block, right=of pe-sum, align=center] (plattform-controller) {Plattform\\ Regler};
    \node [input, below left=2mm and 5mm of plattform-controller.west] (plattform-controller-in2) {};
    \node [sum, right=of plattform-controller] (feedforward-sum) {$-$};
    % Robot Model
    \node [block, above=of plattform-controller, align=center] (robot-model) {kinematische\\ Pr\"adiktion};
    \node [input, above=2mm of robot-model.west] (robot-model-in1) {};
    \node [input, below=2mm of robot-model.west] (robot-model-in2) {};
    \node [tmp, above left=7.5mm and 5mm of robot-model.west] (rc-routing) {};
    % Robot Controller
    \node [block, above=of robot-model, align=center] (robot-controller) {Kinematik\\ Regler};
    \node [joint, right=20mm of robot-controller] (rc-joint) {\textbullet};
    % Robot Input
    \node [sum, left=of robot-controller] (re-sum) {$-$};
    \node [block, left=of re-sum, align=center] (robot-path) {Prozess-\\ Bahnplanung};
    % CMGs incl Controller
    \node [block, right=of feedforward-sum, align=center] (cmg-controller) {CMGs\\ \footnotesize{inkl. Regler}};
    \node [sum, right=of cmg-controller] (torque-sum) {$+$};
    \node [joint, right=20mm of cmg-controller] (cmg-joint) {\textbullet};
    % Robot
    \node [block, above=of cmg-controller, align=center] (robot) {Kinematik\\ z.B. Roboter};
    \node [input, left=5mm of robot.west] (robot-in) {};
    \node [output, above right =3mm and 10mm of robot.east] (robot-out1) {};
    \node [output, above=0mm of robot.east] (robot-out2) {};
    \node [output, above=-3mm of robot.east] (robot-out3) {};
    % PnR Coordinate Model
    \node [block, above=25mm of robot, align=center] (coord-model) {Plattform + Kinematik\\ Bewegungs-Model};
    \node [input, above=0mm of coord-model.east] (coord-model-in1) {};
    \node [input, above=-3mm of coord-model.east] (coord-model-in2) {};
    % Crane, Sensors and Output
    \node [block, right=of torque-sum] (pendulum) {Plattform};
    \node [input, above left=3mm and 5mm of pendulum.west] (pendulum-in1) {};
    \node [input, below left=3mm and 5mm of pendulum.west] (pendulum-in2) {};
    \node [block, below=25mm of pendulum, align=center] (sensor1) {Plattform\\ Sensoren};
    \node [block, above=40mm of pendulum.south, align=center] (sensor2) {Plattform\\ Sensoren};
    \node [joint, right=20mm of pendulum] (output-joint) {\textbullet};
    \node [output, right=15mm of output-joint, node distance=1.5cm] (output) {};
    \node [joint, below right=25mm and 5mm of pe-sum] (pm-joint) {\textbullet};
    \node [joint, below=25mm of feedforward-sum] (pm-joint2) {\textbullet};

% Inertia Estimation Nodes
    \node [block, below=of plattform-controller, align=center] (inertia) {Trägheits-\\ Schätzer};
    \node [input, above=2mm of inertia.east] (inertia-in1) {};
    \node [input, below=2mm of inertia.east] (inertia-in2) {};
% Inertia Estimation Loop
    \draw [->] (inertia) -| (plattform-controller-in2) -- (plattform-controller-in2 -| plattform-controller.west);
    \draw [->] (cmg-joint) |- (inertia-in1);
    \draw [->] (pm-joint2) |- (inertia-in2);

% Crane Nodes
    \node [block, below=20mm of inertia, align=center] (crane-controller) {Kran-\\ Regler};
    \node [input, above=3mm of crane-controller.west] (crane-controller-in1) {};
    \node [block, right=of crane-controller, align=center] (crane-drives) {Kran-\\ Antriebe};
    \node [block, right=of crane-drives, align=center] (crane) {Kran};
    \node [joint, right=20mm of crane] (crane-joint1) {\textbullet};
    \node [sum, left=of crane-controller] (ce-sum) {$-$};
    \node [joint, left=15mm of ce-sum] (crane-joint2) {\textbullet};
    \node [block, left=of ce-sum, align=center] (crane-input) {Kran-\\ Bahnplanung};
    \node [block, below=of crane, align=center] (crane-sensor) {Kran-\\ Sensoren};
%Crane Loop
    \draw [->] (crane-input) -- (crane-joint2) -- node{$K_W$} (ce-sum);
    \draw [->] (ce-sum) -- node[below]{$K_E$} (crane-controller);
    \draw [->] (crane-controller) -- node{$K_S$} (crane-drives);
    \draw [->] (crane-drives) -- (crane);
    \draw [->] (crane) -- (crane-joint1) |- (crane-sensor);
    \draw [->] (crane-sensor) -| node[pos=0.1]{$K_M$} (ce-sum);
    \draw [->] (crane-joint1) -| node[below, pos=0.1]{$K_Y$} (pendulum-in2) -- (pendulum-in2 -| pendulum.west);
    \draw [->] (pm-joint) |- (crane-controller-in1);
    \draw [->] (crane-joint2) -- (orientation-sum);


%%% Paths
% Pendulum Control Loop
    \draw [->] (orientation-input) -- (orientation-sum);
    \draw [->] (orientation-sum) -- node{$P_W$} (pe-sum);
    \draw [->] (pe-sum) -- node{$P_E$} (plattform-controller);
    \draw [->] (plattform-controller) -- (feedforward-sum);
    \draw [->] (feedforward-sum) -- node{$\tau_W$} (cmg-controller);
    \draw [->] (cmg-controller) -- (torque-sum);
    \draw [->] (torque-sum) -- node{$\tau_Y$} (pendulum);
    \draw [->] (pendulum) -- (output-joint) -- node{$P_Y$} (output);
    \draw [->] (output-joint) |- (sensor1);
    \draw [->] (sensor1) -- node[above, pos=0.95]{$P_M$} (pm-joint) -| (pe-sum);
    \draw [->] (output-joint) |- (sensor1);

% Robot Control Loop
    \draw [->] (robot-path) -- node{$R_W$} (re-sum);
    \draw [->] (re-sum) -- node{$R_E$} (robot-controller);
    \draw [->] (robot-controller) -| node[pos=0.2]{$R_S$} (robot-in) -- (robot);
    \draw [->] (rc-joint) |- (rc-routing) |- (robot-model-in1);
    \draw [->] (pm-joint) |- node[pos=0.7, above]{$P_M$} (robot-model-in2);
    \draw [->] (robot-model) -| (feedforward-sum);
    \draw [->] (output-joint) |- (sensor2);
    \draw [->] (sensor2) -- node{$P_M$} (coord-model-in1);
    \draw [->] (coord-model) -| node[pos=0.1, above]{$R_M'$} (re-sum);
    \draw [->] (robot.east |- robot-out1) -- node{$R_M$} (robot-out1) |- (coord-model-in2);
    \draw [->] (robot-out2) -| node[pos=0.4]{$F_Y$} (pendulum-in1) -- (pendulum-in1 -| pendulum.west);
    \draw [->] (robot-out3) -| (torque-sum);

\end{tikzpicture}

\end{document}
